\documentclass[english,12pt]{article}
\usepackage{geometry}
\usepackage{blindtext}
\usepackage{enumerate}
\usepackage{parskip} 
\usepackage{siunitx}
\usepackage{amsmath}
\usepackage{amsfonts}
\usepackage{amssymb}
\usepackage{hyperref}
\usepackage{listings}
\usepackage{inconsolata}
\usepackage{parskip}
\usepackage{graphicx}
\usepackage{wrapfig}
\usepackage{float}
\usepackage{titlesec}
\usepackage{blindtext}
\author{
        Cole Johnson - \texttt{cole.johnson@student.nmt.edu} \\
        Lauren Giles - \texttt{lauren.giles@student.nmt.edu} \\
        Colin Grandjean - \texttt{colin.grandjean@student.nmt.edu} \\
        John Runyon  - \texttt{john.runyon@student.nmt.edu}
}
\title{ \textbf{CSE326: Software Engineering} \\
        Final Project Proposal
}
\begin{document}
\maketitle
\tableofcontents
\section{Introduction}

\subsection{Project Overview and Statement of Proposal}
OCR (Optical Character Recognition) is a technique where an machine attempts
to parse images, or a stream of images, into a matched set of some written
alphabet--often just a set of alphanumeric characters. OCR is used in banking,
note taking applications, and many other services on a daily basis.
One of the most common implementations of OCR is the use of neural networks,
often through supervised learning methods. We propose to create a simple neural
network that will be trained and tested to classify a single image into an
alphanumeric character.

\subsection{Project Scope and Objectives}
The initial scope of the project is the create a simple Optical Character Recognition (OCR)
system using a neural network and employ supervised learning techniques to test and train our
model. Our main objectives for the project include:
\begin{enumerate}[(a.)]
  \item A training and test environment for our neural network
  \item A graphical user interface that the user can draw characters
    for the model to classify them
  \item A graphical user interface to view the created neural networks to
    visually see how each model works
\end{enumerate}

\section{Risk Management Strategy}

\subsection{Risk Table}
\begin{table}[H]
  \begin{center}
    \begin{tabular}[c]{|l|l|l|l|l|}
      \hline
      \multicolumn{1}{|c|}{\textbf{Risks}} & 
      \multicolumn{1}{|c|}{\textbf{Category}} & 
      \multicolumn{1}{|c|}{\textbf{Probability}} &
      \multicolumn{1}{|c|}{\textbf{Impact}} &
      \multicolumn{1}{|c|}{\textbf{RMMM}} \\
      \hline
      a & b & c & d & e\\
      \hline
    \end{tabular}
  \end{center}
\end{table}

\subsection{Discussion of Risks to Be Managed}

\subsection{Risk Mitigation, Monitoring, and Management Plan}

\subsubsection{Risk Mitigation}

\end{document}
