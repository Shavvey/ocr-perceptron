\documentclass[english,12pt]{article}
\usepackage[]{geometry}
\usepackage{tabularx}
\usepackage{blindtext}
\usepackage{enumerate}
\usepackage{parskip} 
\usepackage{siunitx}
\usepackage{amsmath}
\usepackage{amsfonts}
\usepackage{amssymb}
\usepackage{hyperref}
\usepackage{listings}
\usepackage{inconsolata}
\usepackage{parskip}
\usepackage{graphicx}
\usepackage{wrapfig}
\usepackage{float}
\usepackage{titlesec}
\usepackage{blindtext}
\usepackage{multicol}
\usepackage{varwidth}
\usepackage{multirow}
\usepackage{booktabs}
\usepackage[svgnames,table]{xcolor}
\usepackage[tableposition=above]{caption}
\usepackage{pifont}
\usepackage{array,ragged2e}
\newcolumntype{P}[1]{>{\RaggedRight\arraybackslash}p{#1}}
\hfuzz=0.64pt % allow hbox to overflow a little
\author{
        Colin Grandjean - \texttt{colin.grandjean@student.nmt.edu} \\
        Lauren Giles - \texttt{lauren.giles@student.nmt.edu} \\
        Cole Johnson - \texttt{cole.johnson@student.nmt.edu} \\
        John Runyon  - \texttt{john.runyon@student.nmt.edu}
}
\title{ \textbf{CSE326: Software Engineering} \\
        Final Project Proposal
}

\begin{document}
\maketitle
{\footnotesize\tableofcontents} % make a small table of contents on the first page
\pagebreak
\section{Introduction}

\subsection{Project Overview and Statement of Proposal}
OCR (Optical Character Recognition) is a technique where an machine attempts
to parse images, or a stream of images, into a matched set of some written
alphabet--often just a set of alphanumeric characters. OCR is used in banking,
note taking applications, and many other services on a daily basis.
One of the most common implementations of OCR is the use of neural networks,
often through supervised learning methods. We propose to create a simple neural
network that will be trained and tested to classify a single image into an
alphanumeric character.

\subsection{Project Scope and Objectives}
The initial scope of the project is the create a simple Optical Character Recognition (OCR)
system using a neural network and employ supervised learning techniques to test and train our
model. Our main objectives for the project include:
\begin{enumerate}[(a.)]
  \item A training and test environment for our neural network
  \item A graphical user interface that the user can draw characters
    for the model to classify them
  \item A graphical user interface to view the created neural networks to
    visually see how each model works
\end{enumerate}

\section{Risk Management Strategy}

\subsection{Risk Table}
\begin{varwidth}[t]{.4\textwidth}
  \textbf{Category Values}
  \begin{itemize}
    \item \textbf{PS} - Product Size Risk
    \item \textbf{BU} Business Impact Risk
    \item \textbf{CU} - Customer Risk
    \item \textbf{PR} - Process Risk
    \item \textbf{TE} - Technology Risk 
    \item \textbf{DE} - Development Environment Risk
    \item \textbf{ST} - Risk Associated with Staff Size and Experience 
  \end{itemize}
  \end{varwidth}
  \hspace{4em}
  \begin{varwidth}[t]{.5\textwidth}
  \textbf{Impact Values:}
  \begin{itemize}
    \item \textbf{4} - catastrophic
    \item \textbf{3} - critical
    \item \textbf{2} - marginal
    \item \textbf{1} - negligible
  \end{itemize}
\end{varwidth}
\begin{table}[ht]
  \caption{Risk Table}
  \begin{center}
    \begin{tabular}[c]{|P{0.20\linewidth}|P{0.10\linewidth}|P{0.10cm}|P{0.5cm}|P{5.5cm}|}
      \hline
      \multicolumn{1}{|c|}{\textbf{Risks}} & 
      \multicolumn{1}{|c|}{\textbf{Category}} & 
      \multicolumn{1}{|c|}{\textbf{Probability}} &
      \multicolumn{1}{|c|}{\textbf{Impact}} &
      \multicolumn{1}{|c|}{\textbf{RMMM}} \\ [0.5ex]
      \hline\hline
      \small
      Most members on the project do not have a lof of experience with NNs & \textbf{ST} & Moderate & 3 & 
      \small \textbf{Mitigation}
      - Hold meetings where we can review our knowledge of Neural Networks
      and carefully plan how our learning strategy will work and what
      technologies it will involve\\
      \hline
      No robust method for testing whether our model works & \textbf{PR} & High & 2 & 
      \small \textbf{Mitigation}
      - \\
      \hline
    \end{tabular}
  \end{center}
\end{table}

\subsection{Discussion of Risks to Be Managed}

\subsection{Risk Mitigation, Monitoring, and Management Plan}

\subsubsection{Risk Mitigation}

\subsubsection{Risk Monitoring}

\subsubsection{Risk Management (Contingency Plans)}

\section{Schedule}

\subsection{Task List}

\subsection{Timeline Chart}

\subsection{Resource Table}

\section{Project Resources}

\subsection{People}

\subsection{Hardware and Software Resources}

\subsection{Special Resources}

\section{Appendices}

\end{document}
