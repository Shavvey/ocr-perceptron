\documentclass[english,12pt]{article}
\usepackage[]{geometry}
\usepackage{tabularx}
\usepackage{blindtext}
\usepackage{enumerate}
\usepackage{parskip} 
\usepackage{siunitx}
\usepackage{amsmath}
\usepackage{amsfonts}
\usepackage{amssymb}
\usepackage{hyperref}
\usepackage{listings}
\usepackage{inconsolata}
\usepackage{parskip}
\usepackage{graphicx}
\usepackage{wrapfig}
\usepackage{float}
\usepackage{titlesec}
\usepackage{blindtext}
\usepackage{multicol}
\usepackage{varwidth}
\usepackage{multirow}
\usepackage{booktabs}
\usepackage[svgnames,table]{xcolor}
\usepackage[tableposition=above]{caption}
\usepackage{pifont}
\usepackage{array,ragged2e}
\newcolumntype{P}[1]{>{\RaggedRight\arraybackslash}p{#1}}
\hfuzz=0.64pt % allow hbox to overflow a little
\author{
        Colin Grandjean - \texttt{colin.grandjean@student.nmt.edu} \\
        Lauren Giles - \texttt{lauren.giles@student.nmt.edu} \\
        Cole Johnson - \texttt{cole.johnson@student.nmt.edu} \\
        John Runyon  - \texttt{john.runyon@student.nmt.edu}
}
\title{ \textbf{CSE326: Software Engineering} \\
        Final Project Proposal
}

\begin{document}
\maketitle
{\footnotesize\tableofcontents} % make a small table of contents on the first page
\pagebreak
\section{Introduction}

\subsection{Project Overview and Statement of Proposal}
OCR (Optical Character Recognition) is a technique where an machine attempts
to parse images, or a stream of images, into a matched set of some written
alphabet -- often just a set of alphanumeric characters. OCR is used in banking,
note taking applications, and many other services used on a daily basis.
One of the most common implementations of OCR is the use of neural networks,
often through supervised learning methods. We propose to create a simple neural
network that will be trained and tested to classify a single image into an
alphanumeric character using one such supervised learning technique. In
the process of developing the neural network, we'll development a host of tools
to test and view the networks created and some performance metrics to gauge
effectiveness.

\subsection{Project Scope and Objectives}
The initial scope of the project is the create a simple Optical Character Recognition (OCR)
system using a neural network and employ supervised learning techniques to test and train our
model. Our main objectives for the project include:
\begin{enumerate}[(a.)]
  \item build the components of neural network using an object-oriented design principles and programming language
  \item A training and test environment for our neural network
  \item A graphical user interface (GUI) that the user can draw characters
    for the model to classify them
  \item A graphical user interface (GUI) to view the created neural networks to
    visually see how each model works
\end{enumerate}

\section{Risk Management Strategy}

\subsection{Risk Table}
\begin{varwidth}[t]{.5\textwidth}
  \textbf{Category Values}
  \begin{itemize}
    \item \textbf{PS} - Product Size Risk
    \item \textbf{BU} Business Impact Risk
    \item \textbf{CU} - Customer Risk
    \item \textbf{PR} - Process Risk
    \item \textbf{TE} - Technology Risk 
    \item \textbf{DE} - Development Environment Risk
    \item \textbf{ST} - Risk Associated with Staff Size and Experience 
  \end{itemize}
  \end{varwidth}
  \hspace{4em}
  \begin{varwidth}[t]{.5\textwidth}
  \textbf{Impact Values:}
  \begin{itemize}
    \item \textbf{4} - catastrophic
    \item \textbf{3} - critical
    \item \textbf{2} - marginal
    \item \textbf{1} - negligible
  \end{itemize}
\end{varwidth}
\begin{table}[H]
  \caption{Risk Table}
  \begin{center}
    \begin{tabular}[c]{|P{0.20\linewidth}|P{0.10\linewidth}|P{0.10cm}|P{0.5cm}|P{5.5cm}|}
      \hline
      \multicolumn{1}{|c|}{\textbf{Risks}} & 
      \multicolumn{1}{|c|}{\textbf{Category}} & 
      \multicolumn{1}{|c|}{\textbf{Probability}} &
      \multicolumn{1}{|c|}{\textbf{Impact}} &
      \multicolumn{1}{|c|}{\textbf{RMMM}} \\ [0.5ex]
      \hline\hline
      \small
      Most members on the project do not have a lof of experience with NNs & \textbf{ST} & Likely & 3 & 
      \small \textbf{Mitigation}
      - Hold meetings where we can review our knowledge of Neural Networks
      and carefully plan how our learning strategy will work and what
      technologies it will involve\\
      \hline
      No robust method for testing whether our model works & \textbf{PR} & Very Likely & 2 & 
      \small \textbf{Mitigation}
      - Introduce unit tests so that will can, at the very least,
      test the functioning of individual components of the neural network,
      to avoid errors made earlier in the development process.\\
      \hline
      Members on the project use a variety of different hardware and software,
      complicating the development process  & \textbf{DE} & Very Likely  & 1 &  \small \textbf{Mitigation} -
      use a virtual machine to help simplify the process of developing the software, and make
      diagnosing issues easier by sticking to one platform \\
      \hline
    \end{tabular}
  \end{center}
\end{table}

\subsection{Discussion of Risks to Be Managed}
The risk to be managed for this project primarily involve managing the technical complexity,
and deal less with public-facing risks like product size or customer relations. The risks for 
this project require mitigation measures that attempt to alleviate technical risks. For example,
one mitigation effort requires unit testing to help ensure the behavior of each component
of the neural network is functioning as expected.
\subsection{Risk Mitigation, Monitoring, and Management Plan}
\subsubsection{Risk Mitigation}
\begin{enumerate}[1.]
  \item \textbf{No Robust Testing Method to Evaluate the Whole Neural Network - Process Risk:}

  It's very difficult to evaluate how a neural network is flawed, since it's connections
  are often elaborate and difficult to understand. Instead, to help mitigate this risk,
  we propose to introduce unit testing inside the components, and test smaller version
  of the models we will ultimately create.

  \item \textbf{Inexperience in Neural Networks - Staff Risk:}

  If this risk becomes a problem, our mitigation efforts involve
  holding meetings to review relevant information about neural networks. 
  Writing unit tests should also help, since they help the writer understand
  how each component is supposed to function

  \item \textbf{Team Members Use a Variety Different Hardware and Software 
  - Development Environment Risk:}
  
 If different hardware and software environments poses a risk, we can use a virtual machine
 to create a single development environment
\end{enumerate}
\subsubsection{Risk Monitoring}
Risk monitoring involves gauging understanding of each component of the project, ensure
that each part is properly understood and that tests are written properly to ensure each component
is working as intended and understood by members of the project.

\subsubsection{Risk Management (Contingency Plans)}
Contingency plans to reduce risk involve creating and utilizing many different
neural network performance metric, like hamming distances and F1 scores, the ensure
the accuracy of our model and the soundness of our training process. If we encounter
issues implementing the supervised learning process, we can try other techniques instead.
\section{Schedule}

\subsection{Task List}

\subsection{Timeline Chart}

\subsection{Resource Table}

\section{Project Resources}

\subsection{People}
\begin{enumerate}
  \item Colin Grandjean 
  \item Lauren Giles   
  \item Cole Johnson (Team Leader)
  \item John Runyon    
\end{enumerate}
\subsection{Hardware and Software Resources}

\subsection{Special Resources}

\section{Appendices}

\end{document}
